\documentclass[12pt]{article} % размер шрифта

\usepackage{tikz} % картинки в tikz
\usepackage{microtype} % свешивание пунктуации

\usepackage{array} % для столбцов фиксированной ширины

\usepackage{url} % для вставки ссылок \url{...}

\usepackage{indentfirst} % отступ в первом параграфе

\usepackage{sectsty} % для центрирования названий частей
\allsectionsfont{\centering} % приказываем центрировать все sections

\usepackage{amsthm} % теоремы и доказательства

\theoremstyle{definition} % прямой шрифт в условии теорем
\newtheorem{theorem}{Теорема}[section]


\usepackage{amsmath, amssymb} % куча стандартных математических плюшек

\usepackage[top=2cm, left=1.5cm, right=1.5cm, bottom=2cm]{geometry} % размер текста на странице

\usepackage{lastpage} % чтобы узнать номер последней страницы

\usepackage{enumitem} % дополнительные плюшки для списков
%  например \begin{enumerate}[resume] позволяет продолжить нумерацию в новом списке
\usepackage{caption} % подписи к картинкам без плавающего окружения figure


\usepackage{fancyhdr} % весёлые колонтитулы
\pagestyle{fancy}
\lhead{Тигры в вышке}
\chead{}
\rhead{2021-07-02}
\lfoot{}
\cfoot{}
\rfoot{}
\renewcommand{\headrulewidth}{0.4pt}
\renewcommand{\footrulewidth}{0.4pt}



\usepackage{todonotes} % для вставки в документ заметок о том, что осталось сделать
% \todo{Здесь надо коэффициенты исправить}
% \missingfigure{Здесь будет картина Последний день Помпеи}
% команда \listoftodos — печатает все поставленные \todo'шки

\usepackage{booktabs} % красивые таблицы
% заповеди из документации:
% 1. Не используйте вертикальные линии
% 2. Не используйте двойные линии
% 3. Единицы измерения помещайте в шапку таблицы
% 4. Не сокращайте .1 вместо 0.1
% 5. Повторяющееся значение повторяйте, а не говорите "то же"

\usepackage{fontspec} % поддержка разных шрифтов
\usepackage{polyglossia} % поддержка разных языков

\setmainlanguage{russian}
\setotherlanguages{english}

\setmainfont{Linux Libertine O} % выбираем шрифт
% если Linux Libertine не установлен, то
% можно также попробовать Helvetica, Arial, Cambria и т.Д.

% чтобы использовать шрифт Linux Libertine на личном компе,
% его надо предварительно скачать по ссылке
% http://www.linuxlibertine.org/index.php?id=91&L=1

% на сервисах типа sharelatex.com этот шрифт есть :)

\newfontfamily{\cyrillicfonttt}{Linux Libertine O}
% пояснение зачем нужно шаманство с \newfontfamily
% http://tex.stackexchange.com/questions/91507/

\AddEnumerateCounter{\asbuk}{\russian@alph}{щ} % для списков с русскими буквами
\setenumerate[2]{label=\asbuk*)}

%% эконометрические и вероятностные сокращения
\DeclareMathOperator{\Cov}{Cov}
\DeclareMathOperator{\Corr}{Corr}
\DeclareMathOperator{\Var}{Var}
\DeclareMathOperator{\E}{E}
\DeclareMathOperator{\grad}{grad}
\DeclareMathOperator{\plim}{plim}

\newcommand \hb{\hat{\beta}}
\newcommand \hs{\hat{\sigma}}
\newcommand \htheta{\hat{\theta}}
\newcommand \s{\sigma}
\newcommand \hy{\hat{y}}
\newcommand \hY{\hat{Y}}
\newcommand \vunit{\vec{1}}
\newcommand \e{\varepsilon}
\newcommand \he{\hat{\e}}
\newcommand \z{z}
\newcommand \hVar{\widehat{\Var}}
\newcommand \hCorr{\widehat{\Corr}}
\newcommand \hCov{\widehat{\Cov}}
\newcommand \cN{\mathcal{N}}
\let\P\relax
\DeclareMathOperator{\P}{P}




\begin{document}

Ссылка на этот документ: \url{git.io/Jc8uK} или \url{https://github.com/bdemeshev/tigers_lecture_2021-07-02}.


\begin{enumerate}

\item На острове живут 2021 тигр и одна вкусная волшебная антилопа.

Если тигр съест волшебную антилопу, то он сам превратится в волшебную антилопу. 
Мясо волшебной антилопы настолько вкусно, что любой тигр готов ради его вкуса на превращение в антилопу. 
Но ни один тигр не готов быть съеденным ради мяса антилопы. 

Тигры охотятся только в одиночку.

Что будет происходить на этом острове?

\item Кортес с бандой головорезов высадился на берегу. 
Кортес выбирает, нападать ли на деревушку или нет. 
Местная деревушка может либо сразу перейти в подчинение Кортеса, либо принять бой.
Если деревушка примет бой, то выбор появится у Кортеса: либо драться до победного конца, либо
после первых потерь бежать на кораблях обратно. 
Ценность деревушки для Кортеса — одна единица, ценность собственных головорезов — 2 единицы. 
Если Кортес будет драться до конца, то деревушка будет взята, но большинство головорезов погибнет в бою. 
Для жителей деревушки — главное остаться в живых, но и сохранить при этом независимость, конечно, желательно.

Почему Кортес сжёг корабли?


\item Полный золота торговый корабль был захвачен $n \geq 3$ абсолютно рациональными пиратами.
У пиратов есть строгая иерархия: капитан, первый помощник капитана, второй помощник и так далее.

Пираты делят золото так: сначала капитан предлагает свой вариант дележа, затем пираты голосуют за или против.
Делёж одобряется, если за него голосует больше половины пиратов, 
включая предложившего делёж\footnote{Для простоты будем считать, что при равенстве голосов делёж не одобряется, но это не искажает идею решения.}.

Если делёж не одобрен, то капитана убивают, и дележ предлагает первый помощник\ldots

Каждый пират хочет остаться в живых и получить побольше золота. 
При одинаковых выгодах для себя пират голосует за тот вариант, где в живых остается больше сотоварищей.

Золото бесконечно делимо. Какой дележ будет реализован?


\item В гонке за призом в 10 рублей участвуют два игрока, у каждого из которых с собой есть 16
рублей. Они по очереди называют ставки. Ставки должны быть целочисленными, первая —
неотрицательная, каждая последующая ставка — больше предыдущей. Вместо очередной став-
ки любой игрок может сказать «Довольно!» При этом игра оканчивается, оба игрока платят
последние сделанные ими ставки. Приз достаётся игроку с наибольшей последней ставкой.

Найдите равновесие Нэша, совершенное в подыграх.

\newpage
\item Две фирмы одновременно назначают цены на свою продукцию. 
Предельные издержки обеих фирм равны $0.1$.
Рыночный спрос описывается функцией $Q = \max {1 - P, 0}$. 
Весь спросдостается фирме, назначившей наименьшую цену; 
если фирмы назначили одинаковую цену, то спрос делится между ними поровну.

Покупатель полностью осведомлён о ценах. 


\begin{enumerate}
  \item Найдите равновесие Нэша.
  \item Найдите равновесие Нэша, если каждая фирма обязуется вернуть покупателю разницу в цене товара, если конкурент продает дешевле.
\end{enumerate}

\item Собрались $n$ Мудрых тараканов и решили одновременно искать Истину. 
Каждый может добросовестно искать или отдыхать. 
Если Мудрый таракан ищет Истину, то он находит ее независимо от
других с вероятностью $0.5$. 
Если Истина будет найдена хотя бы одним Мудрым тараканом, то он
расскажет ее всем, и все получат полезность $+1$. 
Поиск Истины связан с издержками $0.1$.

\begin{enumerate}
  \item Будет ли одинокий Мудрый таракан искать истину ($n = 1$)?
  \item Найдите равновесие по Нэшу в чистых стратегиях для произвольного $n$;
  \item Найдите симметричное равновесие по Нэшу в смешанных стратегиях для любого $n$;
  \item Как зависит от $n$ доля Мудрых тараканов, ищущих Истину?
  \item Как зависит от n вероятность того, что Истина будет найдена?
\end{enumerate}

\item За окном большого жилого дома драка, плохие парни бьют одного хорошего. 
За дракой в окно наблюдает $n$ обывателей. 
Каждый из обывателей может либо позвонить в полицию, либо просто наблюдать. 

Обыватели хотели бы, чтобы полиция приехала и хороший парень был спасен — в этом случае обыватели получают полезность $1$. 
Однако обыватель не хочет звонить в полицию, звонок означает некоторые издержки равные $0.01$ для звонящего. 
Если в полицию никто не позвонит, то исход будет грустный и у всех обывателей полезность $0$. 

\begin{enumerate}
  \item Найдите равновесие Нэша в чистых стратегиях.
  \item Найдите равновесие Нэша в смешанных стратегиях.
\end{enumerate}


\newpage
\item Три поросёнка, Ниф-Ниф, Нуф-Нуф и Наф-Наф, хотят построить один дом. 
Ниф-Ниф будет доволен любым домом, 
Нуф-Нуф хочет деревянный или более прочный дом,
а Наф-Наф хочет только каменный. 

Стоимости строительства равны: 100 желудей для дома из веточек, 200 — для деревянного и 400 для каменного. 

\begin{enumerate}
  \item Опишите данную задачу как кооперативную игру. 
  \item Найдите вектор Шепли.
  \item Найдите нуклеолус. 
\end{enumerate}


\item Задача о банкротстве. Фирма банкрот должна трём должникам 100, 200 и 300 конфет. 
Однако на счетах банкрота всего 200 конфет. 

\begin{enumerate}
  \item Опишите данную задачу как кооперативную игру. 
  \item Найдите вектор Шепли.
  \item Найдите нуклеолус. 
\end{enumerate}


\item Разберитесь, насколько хорошо концепции из кооперативной теории игр 
помогают определять вклад переменных в качество прогнозов. 

У меня нет ни готового критерия для «хорошо», ни даже примерного представления о том, 
как может выглядеть верный ответ. Раз вектор Шепли оказался полезен, то и другие 
концепции могут «выстрелить» :)


\end{enumerate}


«Задачник для тигров»: \url{https://github.com/bdemeshev/games_pset}



\end{document}