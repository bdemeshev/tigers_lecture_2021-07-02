\documentclass[12pt]{article} % размер шрифта

\usepackage{tikz} % картинки в tikz
\usepackage{microtype} % свешивание пунктуации

\usepackage{array} % для столбцов фиксированной ширины

\usepackage{url} % для вставки ссылок \url{...}

\usepackage{indentfirst} % отступ в первом параграфе

\usepackage{sectsty} % для центрирования названий частей
\allsectionsfont{\centering} % приказываем центрировать все sections

\usepackage{amsthm} % теоремы и доказательства

\theoremstyle{definition} % прямой шрифт в условии теорем
\newtheorem{theorem}{Теорема}[section]


\usepackage{amsmath, amssymb} % куча стандартных математических плюшек

\usepackage[top=2cm, left=1.5cm, right=1.5cm, bottom=2cm]{geometry} % размер текста на странице

\usepackage{lastpage} % чтобы узнать номер последней страницы

\usepackage{enumitem} % дополнительные плюшки для списков
%  например \begin{enumerate}[resume] позволяет продолжить нумерацию в новом списке
\usepackage{caption} % подписи к картинкам без плавающего окружения figure


\usepackage{fancyhdr} % весёлые колонтитулы
\pagestyle{fancy}
\lhead{Тигры в вышке}
\chead{}
\rhead{2021-07-02}
\lfoot{}
\cfoot{}
\rfoot{}
\renewcommand{\headrulewidth}{0.4pt}
\renewcommand{\footrulewidth}{0.4pt}



\usepackage{todonotes} % для вставки в документ заметок о том, что осталось сделать
% \todo{Здесь надо коэффициенты исправить}
% \missingfigure{Здесь будет картина Последний день Помпеи}
% команда \listoftodos — печатает все поставленные \todo'шки

\usepackage{booktabs} % красивые таблицы
% заповеди из документации:
% 1. Не используйте вертикальные линии
% 2. Не используйте двойные линии
% 3. Единицы измерения помещайте в шапку таблицы
% 4. Не сокращайте .1 вместо 0.1
% 5. Повторяющееся значение повторяйте, а не говорите "то же"

\usepackage{fontspec} % поддержка разных шрифтов
\usepackage{polyglossia} % поддержка разных языков

\setmainlanguage{russian}
\setotherlanguages{english}

\setmainfont{Linux Libertine O} % выбираем шрифт
% если Linux Libertine не установлен, то
% можно также попробовать Helvetica, Arial, Cambria и т.Д.

% чтобы использовать шрифт Linux Libertine на личном компе,
% его надо предварительно скачать по ссылке
% http://www.linuxlibertine.org/index.php?id=91&L=1

% на сервисах типа sharelatex.com этот шрифт есть :)

\newfontfamily{\cyrillicfonttt}{Linux Libertine O}
% пояснение зачем нужно шаманство с \newfontfamily
% http://tex.stackexchange.com/questions/91507/

\AddEnumerateCounter{\asbuk}{\russian@alph}{щ} % для списков с русскими буквами
\setenumerate[2]{label=\asbuk*)}

%% эконометрические и вероятностные сокращения
\DeclareMathOperator{\Cov}{Cov}
\DeclareMathOperator{\Corr}{Corr}
\DeclareMathOperator{\Var}{Var}
\DeclareMathOperator{\E}{E}
\DeclareMathOperator{\grad}{grad}
\DeclareMathOperator{\plim}{plim}

\newcommand \hb{\hat{\beta}}
\newcommand \hs{\hat{\sigma}}
\newcommand \htheta{\hat{\theta}}
\newcommand \s{\sigma}
\newcommand \hy{\hat{y}}
\newcommand \hY{\hat{Y}}
\newcommand \vunit{\vec{1}}
\newcommand \e{\varepsilon}
\newcommand \he{\hat{\e}}
\newcommand \z{z}
\newcommand \hVar{\widehat{\Var}}
\newcommand \hCorr{\widehat{\Corr}}
\newcommand \hCov{\widehat{\Cov}}
\newcommand \cN{\mathcal{N}}
\let\P\relax
\DeclareMathOperator{\P}{P}




\begin{document}


\begin{enumerate}
\item На острове живут 2021 тигр и одна вкусная волшебная антилопа.

Если тигр съест волшебную антилопу, то он сам превратится в волшебную антилопу. 
Мясо вол-
шебной антилопы настолько вкусно, что любой тигр готов ради его вкуса на превращение в ан-
тилопу. Но ни один тигр не готов полностью расстаться с жизнью ради мяса антилопы. 
Тигры
охотятся только в одиночку.

Что будет происходить на этом острове?

\item 

\item Задача о банкротстве. Фирма банкрот должна трём должникам 100, 200 и 300 конфет. 
Однако на счетах банкрота всего 200 конфет. 

\begin{enumerate}
  \item Опишите данную задачу как кооперативную игру. 
  \item Найдите вектор Шепли.
  \item Найдите нуклеолус. 
\end{enumerate}

\item Группа из $n$ гномов нашла много золотых слитков в пещере. Начинается обвал, поэтому нужно
срочно убегать из пещеры. После обвала пещера окажется недоступной. Слитки золота тяжелы: в
одиночку ни один гном не может нести слиток, но два гнома могут свободно нести один слиток.
Снаружи пещеры слитки золота можно продать по цене 1 рубль за штуку.

\begin{enumerate}
  \item Найдите ядро и вектор Шепли для произвольного $n$;
  \item Найдите нуклеолус.
\end{enumerate}

\item Разберитесь, насколько хорошо концепции из кооперативной теории игр 
помогают определять вклад переменных в качество прогнозов. 

У меня нет ни готового критерия для «хорошо», ни даже примерного представления о том, 
как может выглядеть верный ответ. Раз вектор Шепли оказался полезен, то и другие 
концепции могут «выстрелить» :)


\end{enumerate}


«Задачник для тигров»: \url{https://github.com/bdemeshev/games_pset}



\end{document}